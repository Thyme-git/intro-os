\documentclass{paper}
\usepackage[UTF8]{ctex}
\usepackage{graphicx}  %插入图片
\usepackage[export]{adjustbox}
\usepackage{float}
\usepackage{geometry}
\usepackage{amssymb}
\usepackage{indentfirst}
% \usepackage{biblatex}
\usepackage[ruled,linesnumbered]{algorithm2e}
\usepackage[colorlinks,linkcolor=blue,bookmarksopen=true,bookmarksnumbered=true]{hyperref}
\usepackage{pythonhighlight}
\usepackage{color}

\setlength{\parindent}{4em}
\setlength{\parskip}{1em}
\geometry{a4paper,scale=0.8}
\SetKwInput{KwIn}{输入}
\SetKwInput{KwOut}{输出}

% \usepackage{listings}
\usepackage{ctex}

% 用来设置附录中代码的样式

\usepackage{listings}
% \usepackage{color}
\usepackage{fontspec}

% 定义Monaco字体
\newfontfamily\Monaco{Monaco}

% 配置lstlisting环境,设置Monaco字体
\lstset{
    language=C,
    basicstyle=\linespread{1.0}\small\Monaco,
    keywordstyle=\color{blue}\bfseries,
    commentstyle=\color{blue},
    stringstyle=\color{purple},
    numbers=left,
    numberstyle=\tiny\Monaco,
    stepnumber=1,
    backgroundcolor=\color{white},
    frame=single,
    frameround=false,
    showspaces=false,
    showstringspaces=false,
    breaklines=true
}

\begin{document}

% 目录
\tableofcontents
\newpage

\section{实验二:Binary Bomb}

\subsection{实验概述}

在本次实验中,我需要使用上课所学的内容拆除一个二进制炸弹(Binary Bomb)。二进制炸弹的拆除过程一共有六个阶段,分别是\verb|phase_1|~\verb|phase_6|。在拆除炸弹的每个阶段,我需要分别输入一个字符串,并且使得在每个阶段中二进制炸弹不会调用\verb|explode_bomb|函数。在本次实验中,拆除炸弹的难度随着每个阶段递增。每个阶段考察的内容如下所示。

\begin{itemize}
    \item 阶段 1:字符串比较
    \item 阶段 2:循环
    \item 阶段 3:条件/分支
    \item 阶段 4:递归调用和栈
    \item 阶段 5:指针
    \item 阶段 6:链表/指针/结构
\end{itemize}

除此之外,本实验还有一个隐藏阶段,需要在阶段四输入特定的字符串进行才会出现。
本实验要求我熟练的掌握和使用GDB调试工具以及OBJDUMP工具。其中GDB调试工具用于调试程序,OBJDUMP工具则用于显示二进制炸弹的反汇编代码。

\subsection{实验内容}

在本次实验中,拆除炸弹的过程主要分为七个阶段,其中第七个阶段是隐藏阶段,将在进行完六个主要阶段后开展。

为了便于后续实验能够顺利地进行,在开展实验之前,我首先需要使用\verb|objdump|工具将可执行文件的反汇编代码保存下来。具体方法是使用如下语句:

\verb|objdump -D ./bomb > ./bomb.s|

使用上述语句即可将反汇编之后输出的结果保存在\verb|bomb.s|文件中了。其中\verb|-D|选项表示将可执行文件中所有的节进行反汇编。

接着我还需要分析实验包中的\verb|bomb.c|文件,便于后续拆除炸弹。\verb|bomb.c|文件主要的代码部分如下所示:

\begin{lstlisting}
input = read_line();
phase_1(input);
phase_defused();
printf("Phase 1 defused. How about the next one?\n");

input = read_line();
phase_2(input);
phase_defused();
printf("That's number 2.  Keep going!\n");

input = read_line();
phase_3(input);
phase_defused();
printf("Halfway there!\n");

input = read_line();
phase_4(input);
phase_defused();
printf("So you got that one.  Try this one.\n");

input = read_line();
phase_5(input);
phase_defused();
printf("Good work!  On to the next...\n");

input = read_line();
phase_6(input);
phase_defused();
\end{lstlisting}

分析上述代码可知,每一个\verb|phase|函数的输入参数都一样,都是一个字符串\verb|input|。而\verb|input|字符串又是\verb|read_line|函数的返回值,即从标准输入中送入程序的一个字符串。要将炸弹拆除,我只需要在六个阶段分别输入相应的字符串即可。

\subsubsection{阶段1 字符串匹配}
\begin{enumerate}
    \item 任务描述
    
            找出\verb|phase_1|中使用的程序中保存的字符并输入相同的字符串以通过本关卡。
    
    \item 实验设计

            在反汇编文件\verb|bomb.s|中查找\verb|phase_1|的汇编代码。找到程序中保存的字符串的地址并用\verb|gdb|打印出相应的字符串。
            
    \item 实验过程

            \begin{enumerate}
                \item 寻找\verb|phase_1|函数的代码并查看字符串的地址
                        
                        在vscode中按下\verb|Ctrl+F|按键,并在弹出的提示框中输入\verb|phase_1|即可定位到\verb|phase_1|的代码段。代码段如下所示:

                        \begin{lstlisting}
08048b33 <phase_1>:
 8048b33:	83 ec 14             	sub    $0x14,%esp
 8048b36:	68 24 a0 04 08       	push   $0x804a024 // 参数:保存的字符串
 8048b3b:	ff 74 24 1c          	push   0x1c(%esp) // 输入的字符串
 8048b3f:	e8 e6 04 00 00       	call   804902a <strings_not_equal>
 8048b44:	83 c4 10             	add    $0x10,%esp
 8048b47:	85 c0                	test   %eax,%eax
 8048b49:	74 05                	je     8048b50 <phase_1+0x1d>
 8048b4b:	e8 d1 05 00 00       	call   8049121 <explode_bomb>
 8048b50:	83 c4 0c             	add    $0xc,%esp
 8048b53:	c3                   	ret
                        \end{lstlisting}

                        函数的第一行\verb|sub $0x14,%esp|首先为\verb|phase_1|分配了\verb|0x14|的栈帧空间。此时\verb|%esp+0x14|即是函数的返回地址,而\verb|%esp+0x18|则是\verb|phase_1|函数的输入,即\verb|main.c|文件中看到的\verb|input|参数。

                        在函数的第二行中\verb|push $0x804a02|将保存的字符串地址压入栈中,作为\verb|strings_not_equal|函数的一个参数。此时\verb|%esp|的值减少了了\verb|0x4|,\verb|input|的地址变为\verb|%esp+0x18+0x4 = %esp+0x1c|。

                        接着,在函数的第三行中,\verb|push 0x1c(%esp)|将\verb|input|压入栈中,作为\verb|strings_not_equal|函数的另一个参数。

                \item 使用\verb|gdb|调试程序,并查看\verb|0x804a024|地址下字符串的值。
                        
                        首先使用以下命令进入\verb|gdb|交互模式:

                        \verb|gdb ./bomb|

                        接着使用以下命令查看\verb|0x804a024|地址下字符串的值:

                        \begin{lstlisting}
(gdb) x /s 0x804a024
0x804a024:      "I am just a renegade hockey mom."
                        \end{lstlisting}

                        由\verb|gdb|输出的结果可知,"I am just a renegade hockey mom."即是我们需要输入的字符串。

            \end{enumerate}

    \item 实验结果
            
            将上述字符串通输入到\verb|ans.txt|中并运行程序,通过了第一个关卡。

            \begin{lstlisting}
$ echo "I am just a renegade hockey mom." >> ans.txt
$ ./bomb ans.txt
Welcome to my fiendish little bomb. You have 6 phases with
which to blow yourself up. Have a nice day!
Phase 1 defused. How about the next one?
            \end{lstlisting}

\end{enumerate}

\subsubsection{阶段2 循环结构}
\begin{enumerate}
\item 任务描述

分析\verb|phase_2|代码,并从循环结构中分析出需要输入的数字以破解本关卡。

\item 实验设计

本阶段实验主要分为以下几个步骤:

\begin{enumerate}
\item 找出需要输入的数字个数;
\item 找到数字存放的位置;
\item 找出所需要输入的数字具体的值。 
\end{enumerate}

\item 实验过程

\begin{enumerate}
\item 找出需要输入的数字个数

查看\verb|phase_2|反汇编代码可以发现以下用于读取数字的函数\verb|read_six_numbers|,相关代码如下所示:

\begin{lstlisting}[label={Read},caption={Read}]
8048b6e:	e8 d3 05 00 00       	call   8049146 <read_six_numbers>
8048b73:	83 c4 10             	add    $0x10,%esp
8048b76:	83 7c 24 04 01       	cmpl   $0x1,0x4(%esp)
\end{lstlisting}

通过函数的名称很容易知道我们需要输入的数字个数是\verb|6|个。

\item 找到数字存放的位置

在\verb|read_six_numbers|函数返回后,可以发现,在代码\ref{Read}中的地址\verb|0x8048b76|处将\verb|0x4(%esp)|与\verb|0x1|作比较,因此我们可以合理推测出所读入的数字存放在\verb|0x4+%esp|附近。

接着使用\verb|gdb|验证上述猜想:

\begin{lstlisting}[language=C]
$ gdb ./bomb
(gdb) b *0x8048b76 // 上述代码中的cmpl 0x1, 0x4(%esp)语句处设置断点
Breakpoint 1 at 0x8048b76
(gdb) r ans.txt // ans中已经保存了第一关的答案
Welcome to my fiendish little bomb. You have 6 phases with
which to blow yourself up. Have a nice day!
Phase 1 defused. How about the next one?
1 1 4 5 1 4 // 第二关的输入测试

Breakpoint 1, 0x08048b76 in phase_2 ()
(gdb) x /6uw 0x4+$esp // 通过观察0x4+$esp中的内容
0xffffc954:     1       1       4       5
0xffffc964:     1       4
\end{lstlisting}

通过观察\verb|0x4+$esp|中的内容可以发现,我们输入的数字存放在以\verb|0x4+$esp|为首地址的连续内存中。

\item 找出所需要输入的数字具体的值 

接着分析代码段,找出第一个数字的值:

\begin{lstlisting}
8048b76:	83 7c 24 04 01       	cmpl   $0x1,0x4(%esp) // 第一个数字
8048b7b:	74 05                	je     8048b82 <phase_2+0x2e>
8048b7d:	e8 9f 05 00 00       	call   8049121 <explode_bomb>
8048b82:	8d 5c 24 04          	lea    0x4(%esp),%ebx
\end{lstlisting}

上述代码段的逻辑十分简单,即:若第一个数字等于\verb|0x1|则跳过\verb|explode_bomb|函数。因此,我们需要输入的第一个数字是\verb|1|。

分析接下来的循环结构代码,得出剩下数字的值:

\begin{lstlisting}
8048b82:	8d 5c 24 04        lea    0x4(%esp),%ebx // 首地址
8048b86:	8d 74 24 18        lea    0x18(%esp),%esi // 尾地址
8048b8a:	8b 03              mov    (%ebx),%eax // loop start
8048b8c:	01 c0              add    %eax,%eax
8048b8e:	39 43 04           cmp    %eax,0x4(%ebx)
8048b91:	74 05              je     8048b98 <phase_2+0x44>
8048b93:	e8 89 05 00 00     call   8049121 <explode_bomb>
8048b98:	83 c3 04           add    $0x4,%ebx
8048b9b:	39 f3              cmp    %esi,%ebx
8048b9d:	75 eb              jne    8048b8a <phase_2+0x36> // loop end
\end{lstlisting}

由\verb|0x18 = 24 = 6*sizeof(int)|可知,\verb|0x18+%esp|是第六个数字的地址。分析上述代码:进入循环前程序先将数组的首地址存放在\verb|%ebx|中,将数组的尾地址存放在\verb|%esi|中。进入循环后,程序将当前数字存放在\verb|%eax|中,并将\verb|2*%eax|与下一个数字(\verb|0x4(%ebx)|)进行比较,若两者相等,则跳过\verb|explode_bomb|。因此剩下的数字的值分别是前一个数字的两倍。

综合上述分析可知,由于第一个数字是\verb|1|,因此接下来的每一个数字分别是\verb|2、4、8、16、32|。

\end{enumerate}

\item 实验结果

将第二关的答案输入\verb|ans.txt|中并运行程序:

\begin{lstlisting}
$ echo "1 2 4 8 16 32" >> ans.txt
$./bomb ans.txt
Welcome to my fiendish little bomb. You have 6 phases with
which to blow yourself up. Have a nice day!
Phase 1 defused. How about the next one?
That's number 2.  Keep going!
\end{lstlisting}

顺利通过!

\end{enumerate}

\subsubsection{阶段3 条件分支}
\begin{enumerate}
\item 任务描述
\item 实验设计
\item 实验过程
\item 实验结果
\end{enumerate}

\subsubsection{阶段4 递归调用}
\begin{enumerate}
\item 任务描述
\item 实验设计
\item 实验过程
\item 实验结果
\end{enumerate}

\subsubsection{阶段5 指针}
\begin{enumerate}
\item 任务描述
\item 实验设计
\item 实验过程
\item 实验结果
\end{enumerate}

\subsubsection{阶段6 链表/指针/结构}
\begin{enumerate}
\item 任务描述
\item 实验设计
\item 实验过程
\item 实验结果
\end{enumerate}

\subsubsection{阶段7 二叉查找树}
\begin{enumerate}
\item 任务描述
\item 实验设计
\item 实验过程
\item 实验结果
\end{enumerate}

\end{document}